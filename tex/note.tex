\documentclass{jpc} %%% last changed 2014-08-20

% JPC Layouting Macros
% THESE ARE ADDED BY THE EDITORIAL TEAM - NO NEED TO SET HERE
%\newcommand{\doisuffix}{v0.i0.999}
% \jpcheading{vol}{issue}{year}{notused}{subm}{publ}{rev}{spec_iss}{title}
%\jpcheading{0}{0}{2000}{}{Mar.~20, 2017}{Jun.~22, 2018}{}{Special issue}
%%% last changed 2014-08-20

%% mandatory lists of keywords
\keywords{census, disclosure avoidance, linkage attack}

%% read in additional TeX-packages or personal macros here:
%% e.g. \usepackage{tikz}
%\usepackage{hyperref}
\usepackage{natbib}
\usepackage[ruled]{algorithm2e}
%%\input{myMacros.tex}
%% define non-standard environments BEYOND the ones already supplied
%% here, for example
\theoremstyle{plain}\newtheorem{satz}[thm]{Satz} %\crefname{satz}{Satz}{S\"atze}
%% Do NOT replace the proclamation environments lready provided by
%% your own.

\def\eg{{\em e.g.}}
\def\cf{{\em cf.}}

%% due to the dependence on amsart.cls, \begin{document} has to occur
%% BEFORE the title and author information:

\begin{document}

\title[Risk of linked census data to transgender families]{The risk of linked census data to transgender families: a simulation study}
%\titlecomment{{\lsuper*}OPTIONAL comment concerning the title, \eg,
%  if a variant or an extended abstract of the paper has appeared elsewhere.}

\author[A.~Flaxman]{Abraham D. Flaxman}	%required
\address{Institute for Health Metrics and Evaluation, University of Washington}	%required
\email{abie@uw.edu}  %optional
%\thanks{thanks 1, optional.}	%optional

\author[O. Keyes]{Os Keyes}	%optional
%\address{address2; addresses should initially be duplicated, even if
%  authors share an affiliation}	%optional
%\email{name2@email2; ditto for email addresses}  %optional
%\thanks{thanks 2, optional.}	%optional

%\author[C.~Name3]{Carla Name3}	%optional
%\address{address 3}	%optional
%\urladdr{name3@url3\quad\rm{(optionally, a web-page can be specified)}}  %optional
%\thanks{thanks 3, optional.}	%optional

%% etc.

%% required for running head on odd and even pages, use suitable
%% abbreviations in case of long titles and many authors:

%%%%%%%%%%%%%%%%%%%%%%%%%%%%%%%%%%%%%%%%%%%%%%%%%%%%%%%%%%%%%%%%%%%%%%%%%%%

%% the abstract has to PRECEDE the command \maketitle:
%% be sure not to issue the \maketitle command twice!

\begin{abstract}
  \noindent Every ten years the United States Census Bureau collects data on all people living in the US, including information on age, sex, race, ethnicity, and household relationship.
  They are required by law to protect this data from disclosure where data provided by any individual can be identified, and, in 2020, they used a novel approach to meet this requirement, the differentially private TopDown Algorithm.
  
  We conducted a simulation study to investigate the risk of disclosing a change in how an individual's sex was recorded in successive censuses.  In a simulated population based on a reconstruction of the 2010 decennial census of Texas, we compared the number of transgender individuals identified by linking simulated census data from 2010 and 2020 under alternative approaches to disclosure avoidance, including swapping in 2020 (as used in the 2010) and TDA in 2020 (as planned for the actual release).
  
  We found that without any disclosure avoidance in 2010 or 2020, a reconstruction-abetted linkage attack identified over 500 transgender children.  With 5\% swapping in 2010 and 2020, it identified 485 individuals, an 11\% decrease.  With swapping in 2010 and TDA in 2020, it identified 43 individuals, a 91\% decrease from swapping.
  
  In light of recent laws prohibiting parents from obtaining medical care for their trans children, our results demonstrate the importance of disclosure avoidance for census data, and suggest that the TDA approach planned by USCB is a substantial improvement compared to the previous approach, but still risks disclosing sensitive information.
\end{abstract}

\maketitle

%% start the paper here:
\section*{Introduction}\label{S:one}

As part of the 2020 decennial census, the US Census Bureau has developed a new approach to disclosure avoidance, based on differential privacy, called the TopDown Algorithm (TDA) (\cite{abowd2019census}).  The details of their approach have been refined iteratively since they first debuted as part of the 2018 end-to-end test (\cite{garfinkel2019end}).  The release of the Demographics and Housing Characteristics (DHC) data in August, 2023 will be the next application of TDA for a data product from the 2020 decennial census. At this time of writing (May 2022) we have the products of the first application of TDA (the Public Law 94-171 redistricting data, released in August, 2021) as well as a demonstration DHC product from a test run in March 2022 (\cite{census2022demonstration}) to help us understand plans and tradeoffs for some of the TDA options previously enumerated  (\cite{petti2019differential}).

In support of their work to develop and validate TDA,  the Census Bureau has previously released a series of Privacy-Protected Microdata Files (PPMFs) by applying iterations of TDA to the 2010 Census Edited File.  The DHC product from March 2022 diverges from this pattern and provides summary tables without releasing a corresponding PPMF.  This invites the question of whether the release of a PPMF or reconstruction of microdata from DHC tables might compromise privacy.  In this work, we investigated empirically how well TDA protects against disclosure of sensitive information on an individual's gender identity in DHC data.


Past investigations of demonstration products have focused primarily on the impact of TDA on accuracy of key census-derived statistics,
%[[refs go here one day]]
and we agree that there are broad, political implications behind statistical accuracy; the framing of census data informs everything from the shape and number of legislative districts to funding and resourcing for minority groups (see, for example, \cite{thompson2012making}). But this is also true of privacy---accurate representation is not an unalloyed good. For many groups, particularly those who are vulnerable to and have experienced active discrimination by state entities, higher accuracy can also mean higher identifiability and higher \textit{scrutiny}. An example of this is undocumented immigrants' relation to questions about citizenship---questions that can be used to identify, surveil, and punish people who are undocumented, and consequently lead to reduced engagement with and trust of the census (see \cite{barreto2019}). More recent in the public eye (although just as longstanding, as highlighted by \cite{canaday2009straight}) are questions of gender (\cite{singer2015profusion}), on which this investigation is focused.

The last few years have seen heightened scrutiny of transgender people (henceforth ``trans''), with a particular focus on (and moral panic around) trans children (see \cite{slothouber2020trans}). This has included actions by state actors to simultaneously legislate against access to care and equal treatment, and use existing mechanisms of government to punish the children and parents who have become identifiable. Most prominently, the governor of Texas, in \cite{abbottletter}, has directed the state Department of Family and Protective Services to investigate the parents of any trans child who receives gender-affirming medical care. In order to do so, he advocates drawing on existing systems for child and parent surveillance, including abuse reporting requirements, to identify targets.

As all of this suggests, there are many reasons for us to be cautious around data availability and the pursuit of accuracy as an untrammelled good. While it is beneficial from a statistical perspective, an absence of privacy simultaneously risks both producing real, material harms for the individuals identified, and undermining trust in the census itself and so (paradoxically) reducing the very accuracy that is aimed for. To demonstrate the importance of factoring identifiability into account---and the necessity of an emphasis on disclosure avoidance in census policy---we used simulation to investigate a risk to privacy, by focusing on the risk of disclosing a child's transgender status, through discordant reporting of binary gender in successive censuses.

\section*{Methods}

We used computer simulation to compare the number of trans children who might be identified in a synthetic population under alternative scenarios of disclosure avoidance.  Our approach began with a synthetic population of size and structure similar to the state of Texas, derived from a reconstruction of the US population on April 1, 2010.  Since our focus is on linking youth between the 2010 and 2020 Decennial Censuses, we included simulants from this population who were aged zero to seven and therefore would be under 18 on April 1, 2020.  We augmented this reconstruction by assigning the simulant's gender based on responses to the Sexual Orientation and Gender Identity (SOGI) module of the Behavioral Risk Factors Surveillance System (BRFSS) fielded in 2019 (\cite{brfss2019}). % [[Described in detail in the appendix]]

We initialized each simulant with attributes for age, gender, race, ethnicity, and household, where age was an integer value representing the age in years, gender was a five-valued variable (with values of transgender boy; transgender girl; transgender, gender nonconforming; cisgender boy; and cisgender girl), race was a 63-valued variable encoding the possible combinations of the five Census racial categories, ethnicity was a 2-valued variable for Hispanic/non-Hispanic, and household was an identifier that encoded census geography (state, county, tract, block) as well as housing unit id.

From this initial population, we simulated the progression of time and data captured in the 2010 and 2020 census as follows: we recorded the age at initialization precisely for each simulant's reported age in the 2010 census, and the age plus 10 for each simulant's reported age in the 2020 census.  We used a simple model of the other key demographic factors of births, deaths, in-migration, and out-migration to simulate how this population might change over the next decade. Since our interest is linking between censuses, we focused on migration, and posited that every household might move, making it harder to link. To realize this, we selected households to stay unmoved from 2010 to 2020 independently, with probability $p_{\text{stay}} = 23\%$ derived from the American Communities Survey (ACS), and updated the 2020 address of each non-staying household by selecting a household uniformly at random from all synthetic households in Texas on Census Day 2010.

Finally, we simulated the reported value of sex on the 2010 and 2020 Decennial Census. Our model starts from the assumption---uncertain though it is---that nearly all of the transgender youth aged zero to seven had their sex reported based on gender-assigned-at-birth in the 2010 Census, and that, for some of the simulants with transgender identities, this would lead to differing responses in the 2020 Census.
Based on this premise, we simulated entries according to the following cases: for cisgender boy simulants, we recorded their sex as male in 2010 and 2020, and similarly for cisgender girl simulants we recorded female.  For transgender boy simulants, we recorded their sex as female in 2010 and recorded their sex with a value chosen uniformly at random from the set $\{\text{male}, \text{female}\}$ in 2020.  Similarly for transgender girl simulants, we recorded their sex as male in 2010 and with a value of female in 2020 with probability 50\%.  For transgender, gender nonconforming simulants we recorded their sex as the same value in 2010 and 2020, with the value chosen uniformly at random from the set $\{\text{male}, \text{female}\}$.

We recorded race and ethnicity identically in 2010 and 2020, matching the value of the simulant's race and ethnicity attributes.

We compared four alternative scenarios of disclosure avoidance: (1) extreme disclosure where names were published, allowing even households that moved to be linked between censuses; (2) tables with no disclosure avoidance, where names were not published, but there was no effort to swap or otherwise perturb the data in published tables; (3) disclosure avoidance by swapping, where 5\% of households were exchanged with another household to protect privacy; and (4) differentially private disclosure avoidance, where the new TDA approach was used to protect against disclosure in published tables.  We now describe our method of quantifying how many transgender simulants would have their gender identity revealed in each of these scenarios.

\emph{Extreme disclosure (Scenario 1):} In this scenario, we assumed that linking on name, age, race, and ethnicity would be able to identify nearly all simulants with discordantly reported values for sex in the 2010 and 2020 censuses.  We therefore counted all simulants with differing values reported for sex in 2010 and 2020 to estimate the number of trans youth who would have their gender identity revealed if census microdata including names were released.  We hypothesized that this would total in the thousands or perhaps even tens of thousands.

\emph{No disclosure avoidance (Scenario 2):} In this scenario, we assumed that only simulants who had a unique combination of age, race, ethnicity, and geography were at risk of having their gender identity revealed by a reconstructed-abetted linkage attack.  Furthermore, we assumed that individuals who moved between the 2010 and 2020 censuses would not have their transgender status revealed and even individuals who were exposed by a unique combination of attributes in 2010 and did not move by 2020 \emph{might} not have their transgender status revealed, if in-migration to their census block resulted in them no longer having a unique combination of attributes in 2020.  We therefore identified all simulants who did not move and had a unique combination of attributes in 2010 and also  in 2020, and counted the simulants in this group with differing values reported for sex in 2010 and 2020.  This constituted our estimate of the number of trans youth who would have their gender identity revealed by a reconstruction-abetted linkage attack if the tables used for reconstruction were published with no disclosure avoidance measures.
We hypothesized that this would total in the hundreds to thousands.

\emph{Swapping for disclosure avoidance (Scenario 3):} We approached this scenario similarly to Scenario 2, but instead of using each simulant's geography directly in the reconstruction-abetted linkage attack, we first chose a random subset of simulants to have a reported location that differed from their true location.  
We achieved this with a simple model analogous to the model of migration described above, where we selected some households to report in a location that is not their actual location independently, with probability $p_\text{swap} = 5\%$ (we chose this value as a modeling assumption broadly aligned with the  publicly available information about the Census Bureau's approach to disclosure avoidance in the 2010 Decennial Census).  For each of the selected households, we chose a reported location by selecting a household uniformly at random from all synthetic households in Texas on Census Day 2010.

We then identified all simulants who did not appear to have moved, according to their (possibly swapped) reported location in the 2010 and 2020 censuses, who had a unique combination of age, race, ethnicity, and geography attributes recorded in both censuses, and counted the simulants in this group with differing values reported for sex in 2010 and 2020.  This constituted our estimate of the number of trans youth who would have their gender identity revealed by a reconstruction-abetted linkage attack if the tables used for reconstruction were protected by swapping.  We hypothesized that this would total would be five to 10\% lower than the total from the no-disclosure-avoidance scenario, and therefore also reveal sensitive information about hundreds or thousands of trans youth.

\emph{TDA for disclosure avoidance (Scenario 4):} Due to time constraints, we were not able to approach this scenario in a way as analogous to Scenarios 1-3 as we would have preferred.  With more time or computer savvy, we would have run TDA ourselves on the synthetic data, after simulating forward ten years.  Instead, we used the Census Bureau's DHC demonstration product to generate our estimate of the risk of a reconstruction-abetted linkage attack in this scenario, which is a little more complicated to explain than the previous three scenarios.

We began with a reconstruction exercise, to come up with a reconstructed microdata file (ReMF) consisting of a row for each reconstructed individual and columns for the attributes of age, sex, race, ethnicity, and geography that was consistent with the tables from the demonstration DHC product for individuals age zero to 17.  We similarly generated an ReMF from the corresponding SF1 tables published as part of the 2010 Decennial Census.  Instead of initializing our synthetic population in 2010 and simulating the progression of time, we initialized our synthetic population in 2020, based on the individuals aged 10 to 17 in the SF1 ReMF.  We then simulated the \emph{regression} of time, going backwards from 2020 to the 2010 census day, when each simulant would be 10 years younger.  We applied our migration model to keep the location in 2010 identical to that in 2020 for only a random fraction simulants, governed again by the parameter $p_{\text{stay}}$.

As in the other scenarios, we endowed each simulant with a gender attribute, which we calibrated to match to measurements from the 2019 BRFSS SOGI module. However, in this scenario, we first set the reported sex in 2020 to match that in the SF1 ReMF, and then set the gender attribute and reported sex in 2010 conditional on the reported sex in 2020.  This allowed us to use the demonstration DHC as our proxy for the privacy afforded by TDA in 2020 in our assessment of the number of trans youth who would have their gender identity revealed by a reconstruction-abetted linkage attack using data protected by swapping in 2010 and TDA in 2020.

To complete this approach, we identified all simulants who had a unique combination of age, race, ethnicity, and geography attributes recorded in 2010, and identified which of these simulants matched a unique individual aged 10 years older in the DHC ReMF.  For each of these simulants, we then compared the reported sex in the 2010 census with the reported sex in the 2020 census. We counted how many of these links were for simulants who were trans youth.  We hypothesized that this would be at least an order of magnitude smaller than the total from the swapping-for-disclosure-avoidance scenario.

\section*{Results}

Our synthetic population included 25,145,561 individual simulants, matching exactly the 2010 population count for Texas.  We focused on the simulants aged zero to seven on April 1, 2010, of which we had 3,095,857.
%%% Could add something about average household size or average number and age of children in households by race/ethnicity.
Among these simulants, 0.53\% were trans, with 0.18\% trans boys, 0.23\% trans girls, and 0.12\% gender nonconforming.
Over the ten years simulation the majority of households moved at least once, and only 23\% of simulants resided in the same census block in 2010 and 2020.

We found that in our scenario with extreme disclosure, where individual-level with linkable names was published (Scenario 1), linking between 2010 and 2020 census data to identify individuals with discordantly reported values for sex would identify over 6,000 trans kids, accounting for 39\% of all trans kids in our simulated version of Texas.

In our scenario where tables like those in SF1 or DHC were published precisely as enumerated, without any disclosure avoidance measures applied (Scenario 2), we found that migration and non-uniqueness substantially reduces the number of trans kids who's gender identity is revealed.  However, there are still 667,072 individuals who are uniquely identified by the age, race, ethnicity, and location in 2010 and 268,492 of them do not move and are still identified uniquely in 2020.  In our simulation, a reconstruction-abetted linkage attack in this scenario still identified over 500 trans kids.

In our next scenario (Scenario 3), we added swapping-based disclosure avoidance to the tables in Scenario 2, and we found that with respect to a reconstructed-abetted linkage attack, swapping functions similarly to migration to prevent identifying trans kids.  At the 5\% swapping level we used in Scenario 3, we found that a reconstruction-abetted linkage attack identified 485 trans kids, an 11\% reduction from the number identified in Scenario 2.

Our final scenario is the closest we considered to the approach proposed by Census Bureau in the most recently released demonstration product. In this scenario, we considered protecting the tables released from the 2010 census with swapping and the tables from the 2020 census with TDA (Scenario 4).  We found that this afforded substantially more protection than the other scenarios we considered.  Because of the alternative route we took to constructing this scenario, we used a larger population for the 2020 census, starting with 3,780,391 simulants ages 10 to 17 on April 1, 2020.  Despite this increase in the total number of simulants to link, we found that TDA was successful in preventing the bulk of the identifications from Scenario 3; in our simulation, a reconstruction-abetted linkage attack identified only 43 trans kids when TDA was used for disclosure avoidance on the 2020 tables, a 91\% reduction in the number identified when swapping was used in Scenario 3.

\section{Discussion}

[[We hope that this convinces some readers of the importance of disclosure avoidance for the 2020 census.]]

[[Past inquiries that compare published results from the 2010 Census to demonstration products have had a problem.  They have all compared the results of swapping to the results of TDA.  This shows that the noise introduced by TDA is different than the noise from swapping, but it doesn't say which is more accurate.  This investigation turns a weakness into a strength, since a reconstruction-abetted linkage attack between 2010 and 2020 Decennial Censuses will be working with some swapped data and some TDA-ed data.  Modeling in a framework like this can also untangle the swapping and TDA issues from other analyses, and simulation is a promising avenue of investigation for other questions.]]

\emph{Limitations}
[[Extensions to migration model: how many households with kids were unmoved ten years later when the census came around again is sure to vary by race/ethnicity, and age or household structure?  I'm sure income is a big part of staying in the same place for 10 years, but that is not in the decennial census.]]

Assumed race and ethnicity were unchanged between 2010 and 2020 censuses. Likely that evolving conception of race and ethnicity led to changes, which would result in some less links in the linkage attack.  We conjecture that this will not substantially change the number identified in our scenarios.

Although the focus of this piece is on trans \textit{children}---specifically,those under-18 in both the 2010 and 2020 census, with different sex records in each---it is worth emphasising that they are not the only people at risk. With the addition of more census tranches (say, 2000, or, going forward, 2030), the range of identifiable people would expand to include trans adults, many of whom, if they have children, are also being targeted for additional scrutiny by state bodies. We should also emphasise several limitations to this note. First, the limited range of sex options on the census means that many trans people whose identities fall outside a simplistic binary do not alter their census markers. Second, we would expect differences in the amount of geographic mobility and consistency in household structure for trans families writ large, with one response to increasing scrutiny, at least for those with means, being to purposefully move states. These limitations suggest this is in fact the \textit{minimal} count of trans people identifiable through the current census approach to data disclosure, and that without changes to the data disclosure approach, well-intentioned efforts to increase the ability of Census Bureau instruments to record and represent trans people (see \cite{whfactsheet}) will only increase the risk of identifiability and harm.

We had to use simulation for this, but it would be great for Census to replicate and expand on analyses such as this one internally, where they can use the real data from census operations that is not possible to share publicly.  The Census Bureau could reproduce this analysis using its internal unprotected data to understand how its implementation differs from this model. We encourage the CB to share with us how much this risk in the true implementation differs from the risk as modeled in this simulation.

 
\section*{Acknowledgment}
  \noindent The authors wish to acknowledge fruitful discussions with danah boyd.

%% in general the use of bibtex is encouraged

\bibliography{note_entries}
\bibliographystyle{abbrvnat}

\end{document}
